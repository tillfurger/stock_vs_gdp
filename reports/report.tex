\documentclass[12pt,a4paper,english]{article}
\usepackage[english]{babel}
\usepackage[utf8]{inputenc}
\usepackage{svg}
\usepackage{amsmath}
\usepackage{johd}
\usepackage{natbib}
\usepackage{booktabs}
\usepackage{authblk}

\title{SPY vs. US GDP}
\author[1]{Martina Stieger}
\author[1]{Flurina Schneider}
%\affil{University of Zurich, Plattenstrasse 14, 8032 Zurich, Switzerland
\author[1]{Till Furger}
\author[1]{Luis Onesimo Leonardo Escobar Farfan}
\affil[1]{University of Zurich, Plattenstrasse 14, 8032 Zurich, Switzerland}

    
%\author{
%    
%    Martina Stieger\thanks{University of Zurich, Plattenstrasse 14, 8032 Zurich, Switzerland, \tt{martina.stieger@uzh.ch}} \rule{0.5in}{0pt}
%
 %   Flurina Schneider\thanks{University of Zurich, Plattenstrasse 14, 8032 Zurich, Switzerland, \tt{flurina.schneider@uzh.ch}} \rule{0.5in}{0pt}
%        
%    Luis Escobar\thanks{University of Zurich, Plattenstrasse 14, 8032 Zurich, Switzerland, \tt{luis.escobar@uzh.ch}} \rule{0.5in}{0pt}
%    
 %   Till Furger\thanks{Department of Business Administration and Wineus AG, University of Zurich, Plattenstrasse 14, 8032 Zurich, Switzerland, \tt{till.furger@uzh.ch}}\\
%    
%    \small \date{\today}
%}


\begin{document}


%%%%%%%%%%%%%%%%%%%%%%%%%%%%%%%%%%%%%%%%%%%%%%%%%%%%%%%%%%%%%%%%%%%%%%%%%%%%%%%%%%%%%%%%%%%%%%%%%%%%%%%%%%
% TITLE PAGE %%%%%%%%%%%%%%%%%%%%%%%%%%%%%%%%%%%%%%%%%%%%%%%%%%%%%%%%%%%%%%%%%%%%%%%%%%%%%%%%%%%%%%%%%%%%%
%%%%%%%%%%%%%%%%%%%%%%%%%%%%%%%%%%%%%%%%%%%%%%%%%%%%%%%%%%%%%%%%%%%%%%%%%%%%%%%%%%%%%%%%%%%%%%%%%%%%%%%%%%

\maketitle

\begin{abstract} 
\noindent A short (up to 250 words) summary of the main contributions of the paper and the context of the research. Full length papers discuss and illustrate methods, challenges, and limitations in the creation, collection, management, access, processing, or analysis of data in humanities research, including standards and formats. This template provides a general outline for full length papers and authors can adapt the headings and include subheadings as they find appropriate.
\end{abstract}

\hfill

\noindent\jel{G1, G11, G14, G17}\\
\noindent\keywords{keyword 1; keyword 2; lower case except names, max 6 }\\

\newpage

%%%%%%%%%%%%%%%%%%%%%%%%%%%%%%%%%%%%%%%%%%%%%%%%%%%%%%%%%%%%%%%%%%%%%%%%%%%%%%%%%%%%%%%%%%%%%%%%%%%%%%%%%%
% BEGIN OF PAPER %%%%%%%%%%%%%%%%%%%%%%%%%%%%%%%%%%%%%%%%%%%%%%%%%%%%%%%%%%%%%%%%%%%%%%%%%%%%%%%%%%%%%%%%%
%%%%%%%%%%%%%%%%%%%%%%%%%%%%%%%%%%%%%%%%%%%%%%%%%%%%%%%%%%%%%%%%%%%%%%%%%%%%%%%%%%%%%%%%%%%%%%%%%%%%%%%%%%

\tableofcontents

\section{Introduction}
Describe the context and motivation of your paper.

\section{Summary statistics}

\begin{figure}[htbp]
	\centering
	\includesvg[width=\textwidth]{figures/histograms}
	\caption{Histograms}
\end{figure}

\begin{figure}[htbp]
	\centering
	\includesvg[width=\textwidth]{figures/boxplots}
	\caption{Boxplots}
\end{figure}




\begin{table}[H]
	
\begin{tabular}{lrrrr}
\toprule
{} &   US GDP &  SPY 500 &  GDP Growth &  SPY Growth \\
\midrule
count &    82.00 &    82.00 &       81.00 &       81.00 \\
mean  &  4128.13 &   160.36 &        0.01 &        0.03 \\
std   &   452.13 &   101.01 &        0.03 &        0.08 \\
min   &  3263.87 &    55.68 &       -0.06 &       -0.22 \\
25\\%   &  3824.75 &    87.52 &       -0.02 &       -0.00 \\
50\\%   &  4036.64 &   112.79 &        0.02 &        0.04 \\
75\\%   &  4487.54 &   211.11 &        0.03 &        0.07 \\
max   &  5110.95 &   469.53 &        0.08 &        0.20 \\
\bottomrule
\end{tabular}

	\caption{Summary Statistics}
\end{table}

\section{Method}
Describe the methods used in the study.

\section{Findings}

\begin{figure}[htbp]
	\centering
	\includesvg[width=\textwidth]{figures/regression}
	\caption{}
\end{figure}

\begin{figure}[htbp]
	\centering
	\includesvg[width=\textwidth]{figures/growth_starting_at_100}
	\caption{}
\end{figure}

\begin{figure}[htbp]
	\centering
	\includesvg[width=\textwidth]{figures/average_growth}
	\caption{}
\end{figure}


\section{Results and discussion}
Describe and discuss the results of the study.


\theendnotes

\bibliography{citation/bibliography.bib}
\bibliographystyle{apalike}

\end{document}